\chapter{Conclusion}
\label{chapter:conclusion}

In this thesis, we proposed a new multimodal optimization technique that aims to break down the multimodal problem into unimodal problems.
This technique is composed of three main components: unimodal detection, region of interest optimization and resource allocation.
We use the three techniques to dynamically isolate potential unimodals and enlarge a specific interesting region on the subspace for algorithms to search more thoroughly.
We also manage the ratio between exploration and exploitation according to the remaining resources,
so that the optimization process can be more efficient.

First, we proposed a unimodal detection technique to find potential \textit{fitness hills} in a real-valued multimodal problem 
by hierarchical clustering and Minimum Description Length.
It depicts the underlying \textit{fitness hills} better than K-Means.
Also unlike K-Means which needs to collaborate with other methods to determine the number of clusters,
our technique intrinsically detects the number of potential unimodals.

Second, we also proposed a method to separate the search space into smaller subspaces in order to enhance performance.
We use linear projection matrix to project the search space to a subspace with well-defined boundaries for feasible solutions.
Therefore, the algorithm only needs to search within a hypercube, constrained by $[0,1]$ in all dimensions.
This technique is preferable for algorithms that requires a box-shape boundaries.
It also creates ROIs on the original search space that isolates the unimodals.
We also use the (1+1)-ES to optimize the projection matrix so that the ROIs have minimal-overlapping, thus enhance the searching efficiency.
By cracking down the original search space into multiple non-overlapping subspace, 
algorithms now only have to search a smaller portion of the landscape which ideally consists of only one unimodal.

Furthermore, we proposed a new Multi-armed Bandit techniques that optimize the resource allocation.
We also explained how to maintain a more stabilize cluster shapes during \textit{recluster}, 
yet still conserve the flexibility to split when needed.
We believe that in the beginning, when there are abundant of resources left, we should invest more resources in exploration.
That means we should update each clusters more equally for exploration.
Later, as the algorithms update the particles, we should be able to merge clusters to eliminate redundant search,
or split a cluster and invest more particles to search in a certain region.
When there are few evaluations left, we should concentrate on exploiting the current best hill.

However, there are still many issues that can be improved.
Our experiment results shows that there are still stability issues.
We need to investigate more on how to delete unnecessary arms.
Thus, this technique tends to cost more evaluations on problems 
that need to move for a long distance after converging to a relatively narrow valley.
Also, the clustering techniques to identify underlying unimodals can be further improved.
The current hierarchical clustering technique can be extended to a complete tree that contains all clusters.
Moreover, since this technique is composed of many relatively complex methods, 
speeding up computational time is also a crucial issue for all hyperparameters tuning techniques.



