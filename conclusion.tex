\chapter{Conclusion}
\label{chapter:conclusion}


%\section{Test Problems} 
Contribution:

find potential \textit{fitness hills} in a real-valued multimodal problem by hierarchical clustering and Minimum Descriptioni Length.
Unlike K-Means clustering that only considers spatial density and requires user-defined number of clusters, or other techniques to help determine the number of clusters, 
the hierarchical clustering process intrinsically captures the unimodal by considering the fitness values.
Later, combining with the MDL, our clustering technique can automatically detect potential unimodals.

define non-overlapping ROIs with well-defined boundaries that isolates and enhance potential unimodals.
This allows the alogrithms to search the a smaller portion of the landscape which should only consists of one unimodal in the projected subspace.
allocate resources

Weakness:
Potentially more NFE on unimodals and moving regions.
Sometimes still unstable.
Requires a better clustering techniques to identify underlying unimodals.


Future work:
A linkage tree attemp to create clusters.
A better way to delete arms
Non-linear transformation




