\chapter{Clustering Techniques}
\label{chapter:clustering}

Describe our basic assumption of function decomposition.
Each subproblem should be composed of an observalbe uni-model.
We wish to identify these uni-models through clustering techniques. 

\section{K-Means clustering}
Describe how K-means clustering works and why it is popular

Describe the limits for K-Means clustering, e.g. it cannot identify density nor unimodality. 

\section{Determine number of clusters}
\subsection{Silhouette coefficient}
Describe how silhouette score decides number of clusters 
\subsection{Gap statistics}
Describe how gap statistics estimates number of clusters.
\subsection{Dip test}
Describe how Dip-test checks unimodality
Describe skynny-dip clustering

\section{Heirarchical Clustering}
Describe the advantage of considering fitness instead of just density.


