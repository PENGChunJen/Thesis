\begin{abstractzh}
本論文提出了一影像中使用者感興趣區域 (region of interest)
偵測之資料集 (benchmark)。
使用者感興趣區域偵測在許多應用中極為有用,
過去雖然有許多使用者感興趣區域之自動偵測演算法被提出,
然而由於缺乏公開資料集,
這些方法往往只測試了各自的小量資料而難以互相比較。
從其它領域可以發現,
基於公開資料集的可重製實驗與該領域突飛猛進密切相關,
因此本論文填補了此領域之不足,
我們提出名為「Photoshoot」的遊戲來蒐集人們對於感興趣區域的標記,
並以這些標記來建立資料集。
透過這個遊戲,我們已蒐集大量使用者對於感興趣區域的標記,
並結合這些資料成為使用者感興趣區域模型。
我們利用這些模型來量化評估五個使用者感興趣區域偵測演算法,
此資料集也可更進一步作為基於學習理論演算法的測試資料,
因此使基於學習理論的偵測演算法成為可能。
\end{abstractzh}

\begin{abstracten}
This thesis presents a new technique for real-valued multi-modal optimization.
Most of the real-world problems can be described as a composition of uni-modal subproblems.
For real-valued optimization, we are interested in problems that are composed of \textit{observable} uni-modals.
However, allocating resources to exploit different uni-modals leads to the 
common delimma between \textit{exploration} and \textit{exploitation}.
This thesis proposed some new techniques that aim to solve multi-modal problems more efficiently.
A new technique combining hierarchical clustering and minimum description length (MDL) 
is proposed to help identify potential uni-modals in the search space.
Then a better defined region of interest (ROI) that contains a subproblem is required 
using a (1+1)-Evolutionary Strategy to optimize a homogeneous linear transform matrix.
Finally, a new multi-armed bandit technique, aiming to maximize the probability of aquiring the global optimium,
is proposed to allocate resources for each uni-modal,
We combined our new techniques with Covariance Matrix Adaptation Evolution Strategy (CMA-ES), 
Standard Particle Swarm Optimization (SPSO) 2011, and Ant Colony Optimization for Continuous Domain (ACO$_R$), 
and evaluate with the CEC2005 Special Session on Real-Parameter Optimization benchmark problems.




detection. ROI detection has many useful applications and many
algorithms have been proposed to automatically detect ROIs.  Unfortunately, due to the lack of benchmarks, these methods were often tested on small data sets that are not available to others,
making fair comparisons of these methods difficult. Examples from
many fields have shown that repeatable experiments using published
benchmarks are crucial to the fast advancement of the fields. To
fill the gap, this thesis presents our design for a collaborative
game, called Photoshoot, to collect human ROI annotations for
constructing an ROI benchmark. With this game, we have gathered a
large number of annotations and fused them into aggregated ROI
models. We use these models to evaluate five ROI detection
algorithms quantitatively. Furthermore, by using the benchmark as
training data, learning-based ROI detection algorithms become
viable.
\end{abstracten}

\begin{comment}
\category{I2.10}{Computing Methodologies}{Artificial Intelligence --
Vision and Scene Understanding} \category{H5.3}{Information
Systems}{Information Interfaces and Presentation (HCI) -- Web-based
Interaction.}

\terms{Design, Human factors, Performance.}

\keywords{Region of interest, Visual attention model, Web-based
games, Benchmarks.}
\end{comment}
